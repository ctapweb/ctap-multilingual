\documentclass{beamer}

\usepackage{listings}
\usepackage{graphicx}

\usetheme{Madrid}

\title[AE Development Conventions]{Conventions for Developing CTAP UIMA AEs}
\author{Xiaobin Chen}
\institute{T\"ubingen University}
\date{\today}

\begin{document}

\frame{\titlepage}

\begin{frame}
\frametitle{Project Setup}
\begin{itemize}
	\item Clone the project \\
		\texttt{git clone
		ssh://username@dn.xiaobin.ch:/var/git-repo/FeatureUIMA.git} 
	\item Import the  project into Eclipse (Maven project)
\end{itemize}
\end{frame}

\begin{frame}
\frametitle{General Structure of the Project}
\begin{description}
	\item [Source Path] \texttt{src/main/java} All Java source packages and
	class source are stored here. 
	\item [Additional Class Path] \texttt{src/main/resources} Everything under
	resources will be copied to the deployment package. 
	\item [Descriptor files] All descriptor files are stored in
	\texttt{resources/annotator} and subdivided into \texttt{annotator} and
	\texttt{featureAE} descriptors. Annotators does only annotation work, while
	fetaureAEs are AEs that calculates feature values. The
	\texttt{descriptor/TAE} folder is used for testing AEs. 
\end{description}
\end{frame}

\begin{frame}
\frametitle{General Structure of the Project}
\begin{description}
	\item [Type system files] Type descriptors are put under
	\texttt{descriptor/type\_system/}. Two general types are defined:
	\texttt{feature\_type} and \texttt{linguistic\_type}. Feature types should
	extend the \texttt{ComplexityFeatureBase} type, which in turn extends UIMA's
	\texttt{AnnotationBase} type and defines an \texttt{id} and a \texttt{value}
	feature. The \texttt{id} feature is used to store the complexity feature's
	id from the database table, while the \texttt{value} feature is used to
	store the calculated feature value, which is of \texttt{Double} type.
	\item [Model files] All language model files should be stored in the
	\texttt{resources/model/} folder. NOTE: the model file name should not
	contain any hyphens (-), because the UIMA Eclipse plugin does not allow
	resource binding with files whose name contains hyphens.
\end{description}
\end{frame}

\begin{frame}
\frametitle{AE Development Notes}
\begin{itemize}
	\item Each AE should override all the three methods: \texttt{initialize()},
	\texttt{process()}, and \texttt{destroy()}. Even if \texttt{initialize()} or
	\texttt{destroy()} do not do anything, it is important to override them for
	easy tracing and debugging.  
	\item Use \texttt{log4j} logging framework. Mark each logging event with
	\texttt{LogMarker.UIMA\_MARKER}. Reusable logging messages should be created
	with a message class in the \texttt{logging.message} package.
	\item Trace each AE method with logging traces.
\end{itemize}
\end{frame}

\begin{frame}
\frametitle{Development Workflow}
	Use the Feature Branch Flow workflow. 
	
	Refer to the workflow tutorial.
\end{frame}

\begin{frame}
\frametitle{The End}
	Enjoy developing CTAP!
\end{frame}
\end{document}
